
\documentclass{article} % For LaTeX2e
\usepackage{iclr2025_conference,times}

% Optional math commands from https://github.com/goodfeli/dlbook_notation.
\input{math_commands.tex}

\usepackage{hyperref}
\usepackage{url}
\usepackage{graphicx} % For including images
\usepackage{booktabs} % For professional-looking tables


\title{INF8245E Fall2024  \\ Kaggle competition \\ Team\_1 }

% Authors must not appear in the submitted version. They should be hidden
% as long as the \iclrfinalcopy macro remains commented out below.
% Non-anonymous submissions will be rejected without review.

\author{Alexis Nicolas - 2143258 \\
\texttt{\ alexis.nicolas@polymtl.ca} \\
\And
Sidney Gharib \\
\texttt{\ sidney.gharib@polymtl.ca} \\
\And
Elizabeth Michaud - 2073093 \\
\texttt{\ elizabeth-1.michaud@polymtl.ca} \\
}

% The \author macro works with any number of authors. There are two commands
% used to separate the names and addresses of multiple authors: \And and \AND.
%
% Using \And between authors leaves it to \LaTeX{} to determine where to break
% the lines. Using \AND forces a linebreak at that point. So, if \LaTeX{}
% puts 3 of 4 authors names on the first line, and the last on the second
% line, try using \AND instead of \And before the third author name.

\newcommand{\fix}{\marginpar{FIX}}
\newcommand{\new}{\marginpar{NEW}}

\iclrfinalcopy % Uncomment for camera-ready version, but NOT for submission.
\begin{document}


\maketitle

\begin{abstract}
This Kaggle explores the relationship between lifestyle factors and the likelihood of diabetes diagnosis.
Utilizing a dataset of 200,000 records with 28 features encompassing demographics,
laboratory test results, and survey responses, we aim to construct predictive models that can identify whether an individual has diabetes.
The research leverages machine learning methodologies to uncover significant patterns and provide actionable insights into the interplay between lifestyle and diabetes risk.
\end{abstract}

\section{Feature design}

The first step was to explore the dataset to understand the kind of data and features we were working with. By doing so, we were 
able to notice that the dataset had more than 200,000 samples and 28 features. While most of the features were numerical 
(binary, discrete and continuous numbers), a few of them were of type "object". 

For each of the discrete and continuous number features, we made a boxplot of their values in the dataset and displayed it. 
This helped us visualize the distribution of the numerical features and their outliers values. The features BMI, MentHlth and PhysHlth
are the features for which there is the most outliers. This is an information the we had to keep in mind while developing our models.

For each of the binary features, we made a plot that showed the distribution of "yes" and "no" labels per possible values which are 0 and 1 since they are binary features. 
When the percentage of "yes" is very similar for 0 and 1, it could mean that the feature in question doesn't have a significant 
impact on the label. Therefore, we considered removing that feature from the model training. 

Since we also wished to explore the features that were non-numerical, we used an ordinal encoding for the features BMI Category, 
Age Group, Education Level and Income Group. We used scikit-learn OrdinalEncoder. Because these features can only take a small amount of different values, we were able to 
specify the order of importance between the different values. For instance, the only possible values for Income Group are "Low Income", 
"Middle Income" and "High Income". It is simple to determine that Low Income is inferior to Middle Income which is inferior to High 
Income. It is important to specify the order, otherwise the ordinal encoder will encode the values in alphabetical order.

Once all the features were encoded, we were able to plot a graph representing the correlation between Diabetes binary 
values from labels.csv and each of the features. We used pandas.DataFrame.corrwith function in order to make such graph. 
The resulting graph shows how much each 28 features tend to affect the diabetes diagnosis. When the result for a feature is positive, 
it means that samples that have a higher value for this feature tends to be labelled as "yes" and vice versa. If the result for a 
feature was very close to zero, we considered removing that feature from the training step. 

We used another metric, called the mutual information score, to measure how much each feature impacts the diagnosis. We computed the mutual 
information score separately for the binary features, the continuous numerical features and the discrete numerical features. 
The lower is the score, the lower should be the impact that this feature has on the diagnosis. This metric was also considered when 
trying to make a good feature selection.

Next, we tried to used the normal probability plot of the features to verify if they were following a Gaussian distribution. It was not 
the case, but if it had been the case, we could potentially have obtained a good classifier by using a Gaussian Naive Bayes model.

Finally, the last method we used in order to select the features to keep was the correlation matrix. By using a correlation matrix, we 
were able to notice if some of the features were highly correlated. If two features are highly correlated, it might not be useful to 
keep both of them to train the model. 

After taking into consideration all of the metrics, we were able to filter out some features from the training step. We decided to 
remove Age Group, MentHlth, HvyAlcoholConsump, NoDocbcCost, Smoker, Fruits, Veggies, Mental Health Risk and AnyHealthcare. 
We tried another combinations of features to remove, but we noticed that this feature selection was the one that gave the best results.
Other features were removed for models where we used PCA. 

In order to complete the data preprocessing, we encoded all the non-numerical features using scikit-learn LabelEncoder, we split the 
dataset into a training set and a validation set and we normalized the training set, the validation set and the testing set using 
scikit-learn StandartScaler. 


\section{Algorithms}

Since we wanted to increase our chances of obtaining a higher score in the Kaggle competition, we trained many different models. We will 
briefly explain the algorithm behind our 5 best models.

Firstly, we used a Random Forest model which is an ensemble learning method that uses multiple decision trees to produce an output. 
It allows to make a better generalization than a single decision tree alone. In order to train all the different decision trees, 
the Random Forest creates a subset of the training set for each decision tree. When we want to make a prediction, each decision tree will 
give its own prediction value and the Random Forest will use the most common prediction value among the decision trees as its prediction value.

Secondly, we used a Decision Tree model with Adaboosting. First of all, a Decision Tree is a model that tries to leaning decision rules from the 
features of the training set. It starts from a root node from where the training set will be split into branches accordingly to the decision rules 
represented by the internal nodes. The leaves of the decision tree represent the class label the tree is going to predict if a sample followed all 
the branches down to that leaf node. Then, we used Adaboosting, an ensemble learning method, on top of our decision tree model. This algorithm trains 
multiple weak learners sequentially, decision stumps in the case of a Decision Tree model, while focussing on the samples that were misclassified. 
Samples that were misclassified will have a more important weight when training the next weak learner. The prediction of this model is given by a 
weighted combination of all learners prediction.

Thirdly, we used a Logistic Regression model which is a probabilistic discriminative model. With the discriminative approach, we want to learn the 
model's parameters directly from the maximum likelihood. The Logistic Regression classifier models the probability of a class by mapping the linear 
combination of the input features with a sigmoid function. For a 2-class classifier, the predicted value is between 0 and 1. If it's greater or equal 
to 0.5, it means the model predicted the positive class for the given sample. Otherwise, the model predicted the negative class. 

Fourthly, we used a Logistic Regression model with Adaboosting. The idea is very similar to Adaboosting with a Decision Tree model, but instead of 
using decision stumps as weak learners, we use a simpler version of a logistic regression.

Fifthly, we used a Neural Network. This model is formed by multiple layers such as an input layer followed by hidden layers that each has its activation 
layer. The last layer is an output layer. Layers have multiple neurons and every neurons make a computation on an input and pass the result to the next 
layer. This is called the forward pass. After, the Neural Network executes a backward pass during which it updates the weights for each neurons. The 
goal is to minimize the error score using gradient descent. 


\section{Methodology}
\label{headings}

First level headings are in small caps,
flush left and in point size 12. One line space before the first level
heading and 1/2~line space after the first level heading.

\subsection{Headings: second level}

Second level headings are in small caps,
flush left and in point size 10. One line space before the second level
heading and 1/2~line space after the second level heading.

\subsubsection{Headings: third level}

Third level headings are in small caps,
flush left and in point size 10. One line space before the third level
heading and 1/2~line space after the third level heading.


\section{Results}

\begin{table}[ht!]
    \centering
    \begin{tabular}{|l|c|}
        \hline
        \textbf{Model} & \textbf{Validation F1 Score} \\ \hline
        Random Forest & 0.4589 \\ \hline
        Random Forest PCA & 0.4309 \\ \hline
        Decision Tree & 0.4252 \\ \hline
        Decision Tree Bagging & 0.4363 \\ \hline
        Decision Tree Adaboosting & 0.4588 \\ \hline
        Logistic Regression & 0.4620 \\ \hline
        Logistic Regression Adaboosting & 0.4582 \\ \hline
        XGBClassifier & 0.2949 \\ \hline
        SVM & 0.4357 \\ \hline
        KNN & 0.2543 \\ \hline
        MLP & 0.2773 \\ \hline
        Neural Network & 0.4706 \\ \hline
    \end{tabular}
    \caption{Validation F1 Scores for Different Models }
    \label{tab:validation-f1-scores}
\end{table}


The results from table 1 are the models validation F1 scores reveal important insights into their performance on the classification task. Among all tested models, the custom Neural Network achieved the highest validation F1 score of 0.4706, demonstrating its effectiveness at capturing complex patterns in the data. This superior performance can be attributed to the carefully designed architecture, which includes multiple layers with dropout regularization, Xavier initialization, and the use of leaky ReLU activations. These design choices appear to enhance the network's ability to generalize, making it the most effective model in this comparison.

The Logistic Regression model, with a validation F1 score of 0.4620, emerged as the second-best performer. Its strong result highlights the model's simplicity and robustness for binary classification tasks. Interestingly, Logistic Regression outperformed ensemble techniques like Adaboosting, which scored slightly lower at 0.4582. This outcome suggests that while boosting often improves base model performance, it may not provide significant gains for Logistic Regression in this dataset.

Random Forest models also performed well, with the standalone Random Forest achieving a validation F1 score of 0.4589. This model's ensemble nature likely helped capture the nuances of the dataset. However, when dimensionality reduction via PCA was applied, the performance dropped to 0.4309. This decline indicates that PCA may have eliminated features essential for classification, reducing the model's effectiveness. Additionally, Decision Tree models, which serve as a base for Random Forest, showed varied results. The single Decision Tree achieved a validation F1 score of 0.4252, which improved modestly with Bagging to 0.4363 and more significantly with Adaboosting to 0.4588. These results illustrate how ensemble techniques help mitigate the inherent overfitting tendencies of decision trees.

On the other hand, some models struggled to achieve competitive performance. For example, the XGBClassifier's validation F1 score of 0.2949 was surprisingly low, given its reputation as a high-performing algorithm. This outcome may indicate that the model requires further hyperparameter tuning or better feature engineering to suit this dataset. Similarly, K-Nearest Neighbors (KNN) scored only 0.2543, likely due to its inability to handle high-dimensional data effectively without feature selection or dimensionality reduction. The Multilayer Perceptron (MLP) model also underperformed, achieving a validation F1 score of 0.2773, suggesting that it may lack the architectural sophistication or training optimizations present in the custom Neural Network.

The Support Vector Machine (SVM), with a validation F1 score of 0.4357, performed comparably to ensemble models like Decision Tree Bagging. This performance reflects the SVM's ability to model nonlinear relationships in the data, though it fell short of the highest-performing methods.

Overall, the results highlight the advantages of more advanced models like Neural Networks while also showcasing the reliability of simpler models such as Logistic Regression and Random Forest. The modest gains achieved by boosting methods indicate that ensemble techniques are valuable but may require careful tuning to unlock their full potential. Future work could explore additional optimizations, such as hyperparameter tuning for XGBClassifier, further experimentation with Neural Network architectures, and the use of ensemble methods that combine predictions from the strongest individual models. 

\begin{table}[ht!]
   \centering
   \begin{tabular}{|c|c|}
       \hline
       \textbf{Submission \#} & \textbf{Public Score} \\ \hline
       1 & 0.449 \\ \hline
       2 & 0.461 \\ \hline
       3 & 0.448 \\ \hline
       4 & 0.226 \\ \hline
       5 & 0.288 \\ \hline
       6 & 0.464 \\ \hline
       7 & 0.462 \\ \hline
       8 & 0.458 \\ \hline
       9 & 0.458 \\ \hline
       10 & 0.455 \\ \hline
       11 & 0.436 \\ \hline
   \end{tabular}
   \caption{Public Scores of Kaggle Submissions}
   \label{tab:kaggle-scores}
\end{table}


In Table 2, we observe a trend in the public scores of our submissions. From Submission 11 to Submission 6, there is a steady increase in scores, reflecting gradual improvements in our models. Submission 6 achieved the highest score of 0.464, closely followed by Submission 7 with 0.462. Submissions 4 and 5, however, represent experiments involving significant changes to the code, such as altering the distribution and composition of the training and validation sets. Specifically, we attempted to train the models using balanced datasets, ensuring equal representation of the classes. However, this approach did not yield positive results, as the test set's composition was fundamentally different. This highlights the challenge of aligning training strategies with the real-world characteristics of the test data.

\begin{table}[ht!]
   \centering
   \begin{tabular}{|l|c|}
       \hline
       \textbf{Hyperparameter} & \textbf{Value} \\ \hline
       Class Weight & \{0: 1, 1: 3\} \\ \hline
       Random State & 0 \\ \hline
       Max Depth & 20 \\ \hline
       Max Leaf Nodes & 200 \\ \hline
       Number of Estimators & 1000 \\ \hline
   \end{tabular}
   \caption{Hyperparameters for the Random Forest Classifier}
   \label{tab:random-forest-hyperparameters}
\end{table}



\begin{table}[ht!]
   \centering
   \begin{tabular}{|l|c|}
       \hline
       \textbf{Component} & \textbf{Details} \\ \hline
       \textbf{Input Size} & \( \text{X\_train\_tensor.shape[1]} \) \\ \hline
       \textbf{Hidden Layers} & 
       \begin{tabular}[c]{@{}l@{}}
           Layer 1: 512 units \\
           Layer 2: 512 units \\
           Layer 3: 256 units \\
           Layer 4: 256 units \\
           Layer 5: 128 units \\
           Layer 6: 128 units \\
           Layer 7: 64 units \\
           Layer 8: 64 units \\
           Layer 9: 32 units \\
           Layer 10: 16 units
       \end{tabular} \\ \hline
       \textbf{Output Layer} & 1 unit (Binary classification) \\ \hline
       \textbf{Activation Function} & Leaky ReLU \\ \hline
       \textbf{Dropout Rate} & 0.1 \\ \hline
       \textbf{Weight Initialization} & Xavier Uniform \\ \hline
       \textbf{Loss Function} & BCEWithLogitsLoss \\ \hline
       \textbf{Optimizer} & NAdam \\ \hline
       \textbf{Learning Rate} & 0.005 \\ \hline
       \textbf{Number of Epochs} & 100 \\ \hline
   \end{tabular}
   \caption{Architecture and Hyperparameters of the Custom Neural Network}
   \label{tab:custom-nn-architecture}
\end{table}

The architecture and hyperparameters of the custom neural network in Table 4 were designed to address a binary classification problem. The model has 10 hidden layers, starting with 512 units in the initial layers and gradually reducing to 16 units in the final hidden layer. This progression allows the network to learn complex patterns while focusing on essential features as it moves deeper. The Leaky ReLU activation function was used to avoid vanishing gradients and handle negative values, ensuring smoother optimization.

A dropout rate of 0.1 was applied after each layer to reduce overfitting by randomly deactivating neurons during training. The Xavier uniform initialization was chosen to distribute weights optimally at the start, helping stabilize gradient flows through the network. The output layer consists of a single unit for binary classification, with BCEWithLogitsLoss as the loss function. This was combined with a positive class weighting factor to handle imbalances in the dataset.

The NAdam optimizer was used for training, combining the benefits of Adam optimization and Nesterov momentum. A learning rate of 0.005 was set to balance the speed of convergence and training stability. The model was trained over 100 epochs, providing enough time to learn the data's patterns without overfitting.


\begin{figure}[ht!]
   \centering
   \includegraphics[width=0.8\textwidth]{results/nn_loss_overtime.png}
   \caption{This is a sample caption for the image.}
   \label{fig:sample-image}
\end{figure}

Figure 1 shows the training and validation loss, as well as the F1 score over 100 epochs. There is a gradual decrease in the training loss until around 30 epochs. The validation loss, on the other hand, remains stable throughout most of the training, with a plateau in the error between epochs 10 and 20. The F1 score consistently stays between 0.40 and 0.47. Therefore, it may be possible to reduce the number of epochs since all metrics converge starting from around 70 epochs.


\section{Discussion}

\section{Statement of Contributions}
\label{others}

I, Sidney Gharib, developed a significant portion of the code for the solution.
I worked on the Random Forest model, the Decision Tree model, the Decision Tree Adaboosting model, a few neural networks and a few other models.
I performed the data analysis with the help of my teammates. Additionally, I contributed to writing the report for the section analyzing the results.

I, Elizabeth Michaud, mainly developed code in the data exploration and preprocessing part. I tried to improve a few of the models that were already 
in place in the code. I tried the stacking model method, but the results were inconclusive. I wrote the feature design and the algorithm sections of 
the report. 

We hereby state that all the work presented in this report is that of the authors.

% Citations within the text should be based on the \texttt{natbib} package
% and include the authors' last names and year (with the ``et~al.'' construct
% for more than two authors). When the authors or the publication are
% included in the sentence, the citation should not be in parenthesis using \verb|\citet{}| (as
% in ``See \citet{Hinton06} for more information.''). Otherwise, the citation
% should be in parenthesis using \verb|\citep{}| (as in ``Deep learning shows promise to make progress
% towards AI~\citep{Bengio+chapter2007}.'').

% The corresponding references are to be listed in alphabetical order of
% authors, in the \textsc{References} section. As to the format of the
% references themselves, any style is acceptable as long as it is used
% consistently.

\subsection{Footnotes}

Indicate footnotes with a number\footnote{Sample of the first footnote} in the
text. Place the footnotes at the bottom of the page on which they appear.
Precede the footnote with a horizontal rule of 2~inches
(12~picas).\footnote{Sample of the second footnote}

\subsection{Figures}

All artwork must be neat, clean, and legible. Lines should be dark
enough for purposes of reproduction; art work should not be
hand-drawn. The figure number and caption always appear after the
figure. Place one line space before the figure caption, and one line
space after the figure. The figure caption is lower case (except for
first word and proper nouns); figures are numbered consecutively.

Make sure the figure caption does not get separated from the figure.
Leave sufficient space to avoid splitting the figure and figure caption.

You may use color figures.
However, it is best for the
figure captions and the paper body to make sense if the paper is printed
either in black/white or in color.
\begin{figure}[h]
\begin{center}
%\framebox[4.0in]{$\;$}
\fbox{\rule[-.5cm]{0cm}{4cm} \rule[-.5cm]{4cm}{0cm}}
\end{center}
\caption{Sample figure caption.}
\end{figure}

\subsection{Tables}

All tables must be centered, neat, clean and legible. Do not use hand-drawn
tables. The table number and title always appear before the table. See
Table~\ref{sample-table}.

Place one line space before the table title, one line space after the table
title, and one line space after the table. The table title must be lower case
(except for first word and proper nouns); tables are numbered consecutively.

\begin{table}[t]
\caption{Sample table title}
\label{sample-table}
\begin{center}
\begin{tabular}{ll}
\multicolumn{1}{c}{\bf PART}  &\multicolumn{1}{c}{\bf DESCRIPTION}
\\ \hline \\
Dendrite         &Input terminal \\
Axon             &Output terminal \\
Soma             &Cell body (contains cell nucleus) \\
\end{tabular}
\end{center}
\end{table}

\section{Default Notation}

In an attempt to encourage standardized notation, we have included the
notation file from the textbook, \textit{Deep Learning}
\cite{goodfellow2016deep} available at
\url{https://github.com/goodfeli/dlbook_notation/}.  Use of this style
is not required and can be disabled by commenting out
\texttt{math\_commands.tex}.


\centerline{\bf Numbers and Arrays}
\bgroup
\def\arraystretch{1.5}
\begin{tabular}{p{1in}p{3.25in}}
$\displaystyle a$ & A scalar (integer or real)\\
$\displaystyle \va$ & A vector\\
$\displaystyle \mA$ & A matrix\\
$\displaystyle \tA$ & A tensor\\
$\displaystyle \mI_n$ & Identity matrix with $n$ rows and $n$ columns\\
$\displaystyle \mI$ & Identity matrix with dimensionality implied by context\\
$\displaystyle \ve^{(i)}$ & Standard basis vector $[0,\dots,0,1,0,\dots,0]$ with a 1 at position $i$\\
$\displaystyle \text{diag}(\va)$ & A square, diagonal matrix with diagonal entries given by $\va$\\
$\displaystyle \ra$ & A scalar random variable\\
$\displaystyle \rva$ & A vector-valued random variable\\
$\displaystyle \rmA$ & A matrix-valued random variable\\
\end{tabular}
\egroup
\vspace{0.25cm}

\centerline{\bf Sets and Graphs}
\bgroup
\def\arraystretch{1.5}

\begin{tabular}{p{1.25in}p{3.25in}}
$\displaystyle \sA$ & A set\\
$\displaystyle \R$ & The set of real numbers \\
$\displaystyle \{0, 1\}$ & The set containing 0 and 1 \\
$\displaystyle \{0, 1, \dots, n \}$ & The set of all integers between $0$ and $n$\\
$\displaystyle [a, b]$ & The real interval including $a$ and $b$\\
$\displaystyle (a, b]$ & The real interval excluding $a$ but including $b$\\
$\displaystyle \sA \backslash \sB$ & Set subtraction, i.e., the set containing the elements of $\sA$ that are not in $\sB$\\
$\displaystyle \gG$ & A graph\\
$\displaystyle \parents_\gG(\ervx_i)$ & The parents of $\ervx_i$ in $\gG$
\end{tabular}
\vspace{0.25cm}


\centerline{\bf Indexing}
\bgroup
\def\arraystretch{1.5}

\begin{tabular}{p{1.25in}p{3.25in}}
$\displaystyle \eva_i$ & Element $i$ of vector $\va$, with indexing starting at 1 \\
$\displaystyle \eva_{-i}$ & All elements of vector $\va$ except for element $i$ \\
$\displaystyle \emA_{i,j}$ & Element $i, j$ of matrix $\mA$ \\
$\displaystyle \mA_{i, :}$ & Row $i$ of matrix $\mA$ \\
$\displaystyle \mA_{:, i}$ & Column $i$ of matrix $\mA$ \\
$\displaystyle \etA_{i, j, k}$ & Element $(i, j, k)$ of a 3-D tensor $\tA$\\
$\displaystyle \tA_{:, :, i}$ & 2-D slice of a 3-D tensor\\
$\displaystyle \erva_i$ & Element $i$ of the random vector $\rva$ \\
\end{tabular}
\egroup
\vspace{0.25cm}


\centerline{\bf Calculus}
\bgroup
\def\arraystretch{1.5}
\begin{tabular}{p{1.25in}p{3.25in}}
% NOTE: the [2ex] on the next line adds extra height to that row of the table.
% Without that command, the fraction on the first line is too tall and collides
% with the fraction on the second line.
$\displaystyle\frac{d y} {d x}$ & Derivative of $y$ with respect to $x$\\ [2ex]
$\displaystyle \frac{\partial y} {\partial x} $ & Partial derivative of $y$ with respect to $x$ \\
$\displaystyle \nabla_\vx y $ & Gradient of $y$ with respect to $\vx$ \\
$\displaystyle \nabla_\mX y $ & Matrix derivatives of $y$ with respect to $\mX$ \\
$\displaystyle \nabla_\tX y $ & Tensor containing derivatives of $y$ with respect to $\tX$ \\
$\displaystyle \frac{\partial f}{\partial \vx} $ & Jacobian matrix $\mJ \in \R^{m\times n}$ of $f: \R^n \rightarrow \R^m$\\
$\displaystyle \nabla_\vx^2 f(\vx)\text{ or }\mH( f)(\vx)$ & The Hessian matrix of $f$ at input point $\vx$\\
$\displaystyle \int f(\vx) d\vx $ & Definite integral over the entire domain of $\vx$ \\
$\displaystyle \int_\sS f(\vx) d\vx$ & Definite integral with respect to $\vx$ over the set $\sS$ \\
\end{tabular}
\egroup
\vspace{0.25cm}

\centerline{\bf Probability and Information Theory}
\bgroup
\def\arraystretch{1.5}
\begin{tabular}{p{1.25in}p{3.25in}}
$\displaystyle P(\ra)$ & A probability distribution over a discrete variable\\
$\displaystyle p(\ra)$ & A probability distribution over a continuous variable, or over
a variable whose type has not been specified\\
$\displaystyle \ra \sim P$ & Random variable $\ra$ has distribution $P$\\% so thing on left of \sim should always be a random variable, with name beginning with \r
$\displaystyle  \E_{\rx\sim P} [ f(x) ]\text{ or } \E f(x)$ & Expectation of $f(x)$ with respect to $P(\rx)$ \\
$\displaystyle \Var(f(x)) $ &  Variance of $f(x)$ under $P(\rx)$ \\
$\displaystyle \Cov(f(x),g(x)) $ & Covariance of $f(x)$ and $g(x)$ under $P(\rx)$\\
$\displaystyle H(\rx) $ & Shannon entropy of the random variable $\rx$\\
$\displaystyle \KL ( P \Vert Q ) $ & Kullback-Leibler divergence of P and Q \\
$\displaystyle \mathcal{N} ( \vx ; \vmu , \mSigma)$ & Gaussian distribution %
over $\vx$ with mean $\vmu$ and covariance $\mSigma$ \\
\end{tabular}
\egroup
\vspace{0.25cm}

\centerline{\bf Functions}
\bgroup
\def\arraystretch{1.5}
\begin{tabular}{p{1.25in}p{3.25in}}
$\displaystyle f: \sA \rightarrow \sB$ & The function $f$ with domain $\sA$ and range $\sB$\\
$\displaystyle f \circ g $ & Composition of the functions $f$ and $g$ \\
  $\displaystyle f(\vx ; \vtheta) $ & A function of $\vx$ parametrized by $\vtheta$.
  (Sometimes we write $f(\vx)$ and omit the argument $\vtheta$ to lighten notation) \\
$\displaystyle \log x$ & Natural logarithm of $x$ \\
$\displaystyle \sigma(x)$ & Logistic sigmoid, $\displaystyle \frac{1} {1 + \exp(-x)}$ \\
$\displaystyle \zeta(x)$ & Softplus, $\log(1 + \exp(x))$ \\
$\displaystyle || \vx ||_p $ & $\normlp$ norm of $\vx$ \\
$\displaystyle || \vx || $ & $\normltwo$ norm of $\vx$ \\
$\displaystyle x^+$ & Positive part of $x$, i.e., $\max(0,x)$\\
$\displaystyle \1_\mathrm{condition}$ & is 1 if the condition is true, 0 otherwise\\
\end{tabular}
\egroup
\vspace{0.25cm}



\section{Final instructions}
Do not change any aspects of the formatting parameters in the style files.
In particular, do not modify the width or length of the rectangle the text
should fit into, and do not change font sizes (except perhaps in the
\textsc{References} section; see below). Please note that pages should be
numbered.

\section{Preparing PostScript or PDF files}

Please prepare PostScript or PDF files with paper size ``US Letter'', and
not, for example, ``A4''. The -t
letter option on dvips will produce US Letter files.

Consider directly generating PDF files using \verb+pdflatex+
(especially if you are a MiKTeX user).
PDF figures must be substituted for EPS figures, however.

Otherwise, please generate your PostScript and PDF files with the following commands:
\begin{verbatim}
dvips mypaper.dvi -t letter -Ppdf -G0 -o mypaper.ps
ps2pdf mypaper.ps mypaper.pdf
\end{verbatim}

\subsection{Margins in LaTeX}

Most of the margin problems come from figures positioned by hand using
\verb+\special+ or other commands. We suggest using the command
\verb+\includegraphics+
from the graphicx package. Always specify the figure width as a multiple of
the line width as in the example below using .eps graphics
\begin{verbatim}
   \usepackage[dvips]{graphicx} ...
   \includegraphics[width=0.8\linewidth]{myfile.eps}
\end{verbatim}
or % Apr 2009 addition
\begin{verbatim}
   \usepackage[pdftex]{graphicx} ...
   \includegraphics[width=0.8\linewidth]{myfile.pdf}
\end{verbatim}
for .pdf graphics.
See section~4.4 in the graphics bundle documentation (\url{http://www.ctan.org/tex-archive/macros/latex/required/graphics/grfguide.ps})

A number of width problems arise when LaTeX cannot properly hyphenate a
line. Please give LaTeX hyphenation hints using the \verb+\-+ command.

\subsubsection*{Author Contributions}
If you'd like to, you may include  a section for author contributions as is done
in many journals. This is optional and at the discretion of the authors.

\subsubsection*{Acknowledgments}
Use unnumbered third level headings for the acknowledgments. All
acknowledgments, including those to funding agencies, go at the end of the paper.


\bibliography{iclr2025_conference}
\bibliographystyle{iclr2025_conference}

\appendix
\section{Appendix}
You may include other additional sections here.


\end{document}
